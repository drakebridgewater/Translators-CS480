\documentclass[letterpaper,10pt]{article}

\usepackage{graphicx}                                        
\usepackage{amssymb}                                         
\usepackage{amsmath}                                         
\usepackage{amsthm}                                          

\usepackage{alltt}                                           
\usepackage{float}
\usepackage{color}
\usepackage{url}

\usepackage{balance}
\usepackage[TABBOTCAP, tight]{subfigure}
\usepackage{enumitem}
\usepackage{pstricks, pst-node}

\usepackage{geometry}
\geometry{textheight=8.5in, textwidth=6in}

\usepackage{tikz}
\usetikzlibrary{arrows,automata}

%----------------------------------------------------------------------------------------
%       Box arround answers use: 
%		\begin{mdframed}[style=MyFrame]
%		\end{mdframed}
%----------------------------------------------------------------------------------------
\usepackage{etex}
\usepackage{tikz}
\usepackage{tikz-qtree}

\usepackage[framemethod=TikZ]{mdframed}
\mdfdefinestyle{MyFrame}{%
    innertopmargin=\baselineskip,
    innerbottommargin=\baselineskip,
    innerrightmargin=20pt,
    innerleftmargin=20pt,
    backgroundcolor=gray!5!white,
    splitbottomskip = 5mm,
    splittopskip = 5mm,
    skipabove=5mm}
    
%----------------------------------------------------------------------------------------
%         Setting for importing code files
%     http://en.wikibooks.org/wiki/LaTeX/Source_Code_Listings
%----------------------------------------------------------------------------------------

\usepackage{import}
\usepackage{listings}
\usepackage{color}

\definecolor{mygreen}{rgb}{0,0.6,0}
\definecolor{mygray}{rgb}{0.5,0.5,0.5}
\definecolor{mymauve}{rgb}{0.58,0,0.82}

\lstset{ %
  backgroundcolor=\color{white},   % choose the background color; you must add \usepackage{color} or \usepackage{xcolor}
  basicstyle=\small,        % the size of the fonts that are used for the code
  breakatwhitespace=false,         % sets if automatic breaks should only happen at whitespace
  breaklines=true,                 % sets automatic line breaking
  captionpos=b,                    % sets the caption-position to bottom
  commentstyle=\color{mygreen},    % comment style
  deletekeywords={...},            % if you want to delete keywords from the given language
  escapeinside={\%*}{*)},          % if you want to add LaTeX within your code
  extendedchars=true,              % lets you use non-ASCII characters; for 8-bits encodings only, does not work with UTF-8
%  frame=single,                    % adds a frame around the code
  keepspaces=true,                 % keeps spaces in text, useful for keeping indentation of code (possibly needs columns=flexible)
  keywordstyle=\color{blue},       % keyword style
  language=Octave,                 % the language of the code
  morekeywords={*,...},            % if you want to add more keywords to the set
  numbers=left,                    % where to put the line-numbers; possible values are (none, left, right)
  numbersep=8pt,                   % how far the line-numbers are from the code
  numberstyle=\tiny\color{mygray}, % the style that is used for the line-numbers
  rulecolor=\color{black},         % if not set, the frame-color may be changed on line-breaks within not-black text (e.g. comments (green here))
  showspaces=false,                % show spaces everywhere adding particular underscores; it overrides 'showstringspaces'
  showstringspaces=false,          % underline spaces within strings only
  showtabs=false,                  % show tabs within strings adding particular underscores
  stepnumber=2,                    % the step between two line-numbers. If it's 1, each line will be numbered
  stringstyle=\color{mymauve},     % string literal style
  tabsize=2,                       % sets default tabsize to 2 spaces
  title=\lstname                   % show the filename of files included with \lstinputlisting; also try caption instead of title
}

%----------------------------------------------------------------------------------------

\newcommand{\toc}{\tableofcontents}

\usepackage{hyperref}

\def\name{Drake Bridgewater }
\def\title{Milestone 3: IBTL Parser }
\def\subtitle{Recursive Descent Parser}
\def\subject{CS }
\def\courseNumber{352 }
\def\courseName{TRANSLATORS }
\def\courseInfo{Winter 2015 }%Class Time: MWF X-X:XX AM}
\def\supervisor{Dr. Jennifer \textsc{Parham-Mocello }} % Supervisor's Name

%% The following metadata will show up in the PDF properties
 \hypersetup{
   colorlinks = false,
   urlcolor = black,
   pdfauthor = {\name},
   pdfkeywords = {\title, \subject, \courseNumber, \courseName, \supervisor},
   pdftitle = {\title},
   pdfsubject = {\subject},
   pdfpagemode = UseNone
 }

\parindent = 0.0 in
\parskip = 0.1 in

\begin{document}


\begin{titlepage}

\newcommand{\HRule}{\rule{\linewidth}{0.5mm}} % Defines a new command for the horizontal lines, change thickness here

\center % Center everything on the page
 
%----------------------------------------------------------------------------------------
%        HEADING SECTIONS
%----------------------------------------------------------------------------------------

\textsc{\LARGE Oregon State University}\\[1.5cm] % Name of your university/college
\textsc{\Large \subject \courseNumber - \courseName}\\[0.5cm] % Major heading such as course name
\textsc{\large \courseInfo}\\[1.5cm] % Minor heading such as course title

%----------------------------------------------------------------------------------------
%        TITLE SECTION
%----------------------------------------------------------------------------------------

\HRule \\[0.4cm]
{ \huge \bfseries \title }\\[0.1cm] % Title of your document
{\small \textit{\textbf{ \subtitle }}}\\[0.2cm]
\HRule \\[7.5cm]
 
%----------------------------------------------------------------------------------------
%        AUTHOR SECTION
%----------------------------------------------------------------------------------------

\begin{minipage}{0.4\textwidth}
\begin{flushleft} \large
\emph{Author:}\\
\name
\end{flushleft}
\end{minipage}
~
\begin{minipage}{0.4\textwidth}
\begin{flushright} \large
\emph{Professor:} \\
\supervisor
\end{flushright}
\end{minipage}\\[4cm]

% If you don't want a supervisor, uncomment the two lines below and remove the section above
%\Large \emph{Author:}\\
%John \textsc{Smith}\\[3cm] % Your name

%----------------------------------------------------------------------------------------
%        DATE SECTION
%----------------------------------------------------------------------------------------

DUE 01/16/15 (11:59pm)\\
{\large \today}\\[3cm] % Date, change the \today to a set date if you want to be precise

%----------------------------------------------------------------------------------------
%        LOGO SECTION
%----------------------------------------------------------------------------------------

%\includegraphics{Logo}\\[1cm] % Include a department/university logo - this will require the graphicx package
 
%----------------------------------------------------------------------------------------

\vfill % Fill the rest of the page with whitespace

\end{titlepage}

\tableofcontents
\vfill % Fill the rest of the page with whitespace
\newpage

\section{Source Code Descriptions}
\begin{mdframed}[style=MyFrame]
The way I approached this problem was one paper with drawing out what how I would perform each of of the operations for a given string. With a few iterations I was able to create the parser that logically followed the grammar, but some modification we needed to account for the left recursion and to factor out the repeated tokens. 
\end{mdframed}
\subsection{node.py}
\begin{mdframed}[style=MyFrame]
Since a tree is just a single node with child nodes I created a node that would allow printing in a familiar format for easy readability 
\end{mdframed}
\subsection{defines.py}
\begin{mdframed}[style=MyFrame]
I decided to place all the global variable into a file for easy manipulation. This fill contains the token ID and also defines what a token is. 
\end{mdframed}
\subsection{myparser.py}
\begin{mdframed}[style=MyFrame]
The bulk of this project was to develop a parser that will spit out a list of tokens in a fashion that would allow seeing scope. \supervisor recommended that we implement it as a tree therefore the node I created. Every time I saw a object in the grammar including 's', 'expr', 'oper', etc. I would create a token and depending on how it is related to its parent it would be added as a child or as a leaf node along side. Once I had this idea I need to come up with a way of documenting my trails to the node for debugging purposes therefore I added a need node each time a function was called and when a function was called within it would be added as a child. 
\end{mdframed}
\subsection{lexer.py}
\begin{mdframed}[style=MyFrame]
The lexer of this assignment was to recognize the chars one at a time and take the one with the longest prefix. This would allow gathering o 
\end{mdframed}

\newpage
\section{Report}
%Handwritten Answers to Milestone Questions:
%Specification (what do you think the purpose of this milestone is)
%Processing (how did you go about solving the problem)
%Testing Requirement (how did you test for correctness)
%Retrospective (what did you learn in this milestone)

\begin{mdframed}[style=MyFrame]
\textbf{The purpose} of this mile stone was to ensure that we can store, retrieve and redistribute in such a way that the grammar is followed. This ensures that we understand the code that we are writing and verifies we understood what we needed to do. Like many people I did have to do some refactoring to the previous milestone to ensure it would work with this one (which took the first week). \textbf{To solve} this problem I looked had to ensure I understood what the input was (milestone 1) and understand what I needed the out put to become. Since I had already developed a tree I had to make sure that I understood what I was needing to do in the middle. This part was just a bunch of if statements but they were intertwined making some operations quite confusing leading me to draw it out on paper multiple time. \textbf{To test} this I started with one of the smallest accepted statements '( )' and then moved on to have a test case for each of the lines in the grammar. Overall I learned python more, as I know understand how to create structure like elements and I have successfully implemented a recursion. This ensured that we knew our recursive algorithms and verified we new how to work with a medium sized project. 
\end{mdframed}



\newpage
\section{Source Code}
\subsection{main.py}
\lstinputlisting[language=Python]{main.py}
\subsection{node.py}
\lstinputlisting[language=Python]{node.py}
\subsection{defines.py}
\lstinputlisting[language=Python]{defines.py}
\subsection{myparser.py}
\lstinputlisting[language=Python]{myparser.py}
\subsection{lexer.py}
\lstinputlisting[language=Python]{lexer.py}



\end{document}