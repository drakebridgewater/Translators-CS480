\documentclass[letterpaper,10pt]{article}

\usepackage{graphicx}                                        
\usepackage{amssymb}                                         
\usepackage{amsmath}                                         
\usepackage{amsthm}                                          

\usepackage{alltt}                                           
\usepackage{float}
\usepackage{color}
\usepackage{url}

\usepackage{balance}
\usepackage[TABBOTCAP, tight]{subfigure}
\usepackage{enumitem}
\usepackage{pstricks, pst-node}

\usepackage{geometry}
\geometry{textheight=8.5in, textwidth=6in}

\usepackage{tikz}
\usetikzlibrary{arrows,automata}

%----------------------------------------------------------------------------------------
%       Box arround answers use: 
%		\begin{mdframed}[style=MyFrame]
%		\end{mdframed}
%----------------------------------------------------------------------------------------
\usepackage{etex}
\usepackage{tikz}
\usepackage{tikz-qtree}

\usepackage[framemethod=TikZ]{mdframed}
\mdfdefinestyle{MyFrame}{%
    innertopmargin=\baselineskip,
    innerbottommargin=\baselineskip,
    innerrightmargin=20pt,
    innerleftmargin=20pt,
    backgroundcolor=gray!5!white,
    splitbottomskip = 5mm,
    splittopskip = 5mm,
    skipabove=5mm}
    
%----------------------------------------------------------------------------------------
%         Setting for importing code files
%     http://en.wikibooks.org/wiki/LaTeX/Source_Code_Listings
%----------------------------------------------------------------------------------------

\usepackage{import}
\usepackage{listings}
\usepackage{color}

\usepackage{verbatim}

\definecolor{mygreen}{rgb}{0,0.6,0}
\definecolor{mygray}{rgb}{0.5,0.5,0.5}
\definecolor{mymauve}{rgb}{0.58,0,0.82}

\lstset{ %
  backgroundcolor=\color{white},   % choose the background color; you must add \usepackage{color} or \usepackage{xcolor}
  basicstyle=\small,        % the size of the fonts that are used for the code
  breakatwhitespace=false,         % sets if automatic breaks should only happen at whitespace
  breaklines=true,                 % sets automatic line breaking
  captionpos=b,                    % sets the caption-position to bottom
  commentstyle=\color{mygreen},    % comment style
  deletekeywords={...},            % if you want to delete keywords from the given language
  escapeinside={\%*}{*)},          % if you want to add LaTeX within your code
  extendedchars=true,              % lets you use non-ASCII characters; for 8-bits encodings only, does not work with UTF-8
%  frame=single,                    % adds a frame around the code
  keepspaces=true,                 % keeps spaces in text, useful for keeping indentation of code (possibly needs columns=flexible)
  keywordstyle=\color{blue},       % keyword style
  language=Octave,                 % the language of the code
  morekeywords={*,...},            % if you want to add more keywords to the set
  numbers=left,                    % where to put the line-numbers; possible values are (none, left, right)
  numbersep=8pt,                   % how far the line-numbers are from the code
  numberstyle=\tiny\color{mygray}, % the style that is used for the line-numbers
  rulecolor=\color{black},         % if not set, the frame-color may be changed on line-breaks within not-black text (e.g. comments (green here))
  showspaces=false,                % show spaces everywhere adding particular underscores; it overrides 'showstringspaces'
  showstringspaces=false,          % underline spaces within strings only
  showtabs=false,                  % show tabs within strings adding particular underscores
  stepnumber=2,                    % the step between two line-numbers. If it's 1, each line will be numbered
  stringstyle=\color{mymauve},     % string literal style
  tabsize=2,                       % sets default tabsize to 2 spaces
  title=\lstname                   % show the filename of files included with \lstinputlisting; also try caption instead of title
}

%----------------------------------------------------------------------------------------

\newcommand{\toc}{\tableofcontents}

\usepackage{hyperref}

\def\name{Drake Bridgewater }
\def\title{Milestone 5: Loops and Variables/Local Values }
\def\subtitle{}
\def\subject{CS }
\def\courseNumber{352 }
\def\courseName{TRANSLATORS }
\def\courseInfo{Winter 2015 }%Class Time: MWF X-X:XX AM}
\def\supervisor{Dr. Jennifer \textsc{Parham-Mocello }} % Supervisor's Name

%% The following metadata will show up in the PDF properties
 \hypersetup{
   colorlinks = false,
   urlcolor = black,
   pdfauthor = {\name},
   pdfkeywords = {\title, \subject, \courseNumber, \courseName, \supervisor},
   pdftitle = {\title},
   pdfsubject = {\subject},
   pdfpagemode = UseNone
 }

\parindent = 0.0 in
\parskip = 0.1 in

\begin{document}


\begin{titlepage}

\newcommand{\HRule}{\rule{\linewidth}{0.5mm}} % Defines a new command for the horizontal lines, change thickness here

\center % Center everything on the page
 
%----------------------------------------------------------------------------------------
%        HEADING SECTIONS
%----------------------------------------------------------------------------------------

\textsc{\LARGE Oregon State University}\\[1.5cm] % Name of your university/college
\textsc{\Large \subject \courseNumber - \courseName}\\[0.5cm] % Major heading such as course name
\textsc{\large \courseInfo}\\[1.5cm] % Minor heading such as course title

%----------------------------------------------------------------------------------------
%        TITLE SECTION
%----------------------------------------------------------------------------------------

\HRule \\[0.4cm]
{ \huge \bfseries \title }\\[0.1cm] % Title of your document
{\small \textit{\textbf{ \subtitle }}}\\[0.2cm]
\HRule \\[7.5cm]
 
%----------------------------------------------------------------------------------------
%        AUTHOR SECTION
%----------------------------------------------------------------------------------------

\begin{minipage}{0.4\textwidth}
\begin{flushleft} \large
\emph{Author:}\\
\name
\end{flushleft}
\end{minipage}
~
\begin{minipage}{0.4\textwidth}
\begin{flushright} \large
\emph{Professor:} \\
\supervisor
\end{flushright}
\end{minipage}\\[4cm]

% If you don't want a supervisor, uncomment the two lines below and remove the section above
%\Large \emph{Author:}\\
%John \textsc{Smith}\\[3cm] % Your name

%----------------------------------------------------------------------------------------
%        DATE SECTION
%----------------------------------------------------------------------------------------

DUE 03/15/15 (11:59pm)\\
{\large \today}\\[3cm] % Date, change the \today to a set date if you want to be precise

%----------------------------------------------------------------------------------------
%        LOGO SECTION
%----------------------------------------------------------------------------------------

%\includegraphics{Logo}\\[1cm] % Include a department/university logo - this will require the graphicx package
 
%----------------------------------------------------------------------------------------

\vfill % Fill the rest of the page with whitespace

\end{titlepage}

\tableofcontents
\vfill % Fill the rest of the page with whitespace
\newpage

\section{Source Code Descriptions}
\begin{mdframed}[style=MyFrame]
The way I approached this problem was one paper with drawing out what how I would perform each of of the operations for a given string. With a few iterations I was able to create the parser that logically followed the grammar, but some modification we needed to account for the left recursion and to factor out the repeated tokens. 
\end{mdframed}
\subsection{node.py}
\begin{mdframed}[style=MyFrame]
Since a tree is just a single node with child nodes I created a node that would allow printing in a familiar format for easy readability 
\end{mdframed}
\subsection{codegen.py}
\begin{mdframed}[style=MyFrame]
This was created to take the tree that we created in the last assignment. With the tree we would walk through it, post-order, and then as we saw the elements we pushed them out to to file. This allowed the code to be rather simple. Once that was finished I converted the variable to gforth code and wrote that to another file.
\end{mdframed}
\subsection{defines.py}
\begin{mdframed}[style=MyFrame]
I decided to place all the global variable into a file for easy manipulation. This fill contains the token ID and also defines what a token is. 
\end{mdframed}
\subsection{myparser.py}
\begin{mdframed}[style=MyFrame]
The bulk of this project was to develop a parser that will spit out a list of tokens in a fashion that would allow seeing scope. \supervisor recommended that we implement it as a tree therefore the node I created. Every time I saw a object in the grammar including 's', 'expr', 'oper', etc. I would create a token and depending on how it is related to its parent it would be added as a child or as a leaf node along side. Once I had this idea I need to come up with a way of documenting my trails to the node for debugging purposes therefore I added a need node each time a function was called and when a function was called within it would be added as a child. 
\end{mdframed}
\subsection{lexer.py}
\begin{mdframed}[style=MyFrame]
The lexer of this assignment was to recognize the chars one at a time and take the one with the longest prefix. This would allow gathering o 
\end{mdframed}

\newpage
\section{Report}
%Handwritten Answers to Milestone Questions:
%Specification (what do you think the purpose of this milestone is)
%Processing (how did you go about solving the problem)
%Testing Requirement (how did you test for correctness)
%Retrospective (what did you learn in this milestone)

\begin{mdframed}[style=MyFrame]
This assignment was much harder then I thought it was going to be but \textbf{the purpose} of this assignment was to ensure that our tree was producing the correct output and that we understand tokens, trees and parsing. To approach this problem I took some single statements and got them to parse all the way through. Once I had those working I was able to do multiple statements per file. To\textbf{solve} this problem I need to understand gforth and my code. Therefore I used my code to figure out how the gforth worked because gforth is difficult to grasp. As I was going I was able to see tokens that needed to be moved and because I was looking for full statements before converting to gforth I was able to reorganize easily.

\end{mdframed}

\subsection{Develop a formal definition of the code generation algorithm} % from the intended (naïve) semantics for IBTL based on the previously developed code generator.

\begin{mdframed}[style=MyFrame]
How I implemented this was probably not the best way but I found it to work in many cases. I chose
to implement this code generation by reading in the tokens from the tree as I was expecting them, post order traversal. This allowed me to see when there was an error as I only expect it to work this way. ie. if I have a let statement then it will be followed by the varlist. Once I saw enough to determine the output of that command I was able to produce the gforth code. Here the idea was to take the token I have seen and reorganize them so that they are in the post fix language, gforth. 
\end{mdframed}

\subsection{Test the resulting program for correctness based on the formal definition}

\begin{mdframed}[style=MyFrame]
This is a very hard concept that I feel OSU needs to focus on more, but my idea was create test cases that contain single statements. Once a statement passed I moved to the next one. Having all the statements implemented before was helpful but I missed may corner cases because I did not spend enough time thinking about the corners and test cases. The next step was combing statements and testing them. 
\end{mdframed}

\newpage
\section{README}
\verbatiminput{README.txt}
\newpage
\section{Source Code}
\subsection{main.py}
\lstinputlisting[language=Python]{main.py}
\subsection{codegen.py}
\lstinputlisting[language=Python]{codegen.py}
\subsection{node.py}
\lstinputlisting[language=Python]{node.py}
\subsection{defines.py}
\lstinputlisting[language=Python]{defines.py}
\subsection{myparser.py}
\lstinputlisting[language=Python]{myparser.py}
\subsection{lexer.py}
\lstinputlisting[language=Python]{lexer.py}



\end{document}